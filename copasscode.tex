\documentclass[a4paper]{article}

%% Language and font encodings
\usepackage[english]{babel}
\usepackage[T1]{fontenc}

%% Sets page size and margins
\usepackage[a4paper,top=3cm,bottom=2cm,left=3cm,right=3cm,marginparwidth=1.75cm]{geometry}

% Useful packages
\usepackage{amsmath}
\usepackage{graphicx}
% \usepackage[colorinlistoftodos]{todonotes}
\usepackage[colorlinks=true, allcolors=blue]{hyperref}

\title{COP290 C LAB: SPREADSHEET PROGRAM}
\author{Samarth Patel, Priyanka Sar, Rishika Garg}

\begin{document}
\maketitle

\begin{abstract}
Your abstract.
\end{abstract}

\section{DESIGN DECISIONS}

Our spreadsheet program consists of a Makefile that compiles our source files: \texttt{asscop.c}, \texttt{display.c}, \texttt{expcalc.c}, \texttt{graph.c}, \texttt{avltree.c}, and \texttt{helper.c} to create an executable spreadsheet-like application. This spreadsheet accepts formulas and control inputs.

\texttt{Display.c} initializes our sheet with the required number of rows and columns, with each cell initialized to zero. Upon receiving input, it calls the \texttt{getline} function (defined in \texttt{helper.c}) to read commands, scroll the sheet, or disable/enable the display. If necessary, it invokes the parser (defined in \texttt{asscop.c}) for assignments and recalculations. 

Each cell in our spreadsheet is a structure containing:
\begin{itemize}
    \item Its current value,
    \item Flags for recalculations,
    \item A visit marker for graph traversal,
    \item References to dependent rows and columns,
    \item Pointers to AVL tree roots for dependency tracking and min/max calculations.
\end{itemize}

We use AVL trees (\texttt{avltree.c}) to efficiently manage dependencies and support operations like MIN and MAX.

\section{EDGE CASES AND ERROR SCENARIOS}

\subsection{Padding for 32-bit Integers}
We added a padding of 15 in \texttt{display.c} to ensure that signed 32-bit integers are displayed correctly.

\subsection{Invalid Range}
Providing invalid ranges in functions such as MAX, MIN, SUM, AVG, and STDEV results in an error. Examples:
\begin{enumerate}
    \item \texttt{A1=AVG(B2:B1)} will output: \textbf{[0.0] (invalid range) >}
    \item \texttt{A1=SUM(A100:F1)} will output: \textbf{[0.0] (invalid range) >}
\end{enumerate}

\subsection{Invalid Commands}
Various invalid input commands are handled:
\begin{enumerate}
    \item Empty range functions such as \texttt{SLEEP()}, \texttt{MAX()}, \texttt{MIN()}, \texttt{AVG()}, \texttt{SUM()} all return \textbf{[0.0] (unrecognized cmd) >}
    \item Missing parentheses or incorrect syntax in functions, e.g., \texttt{A1=MAX(B1:B2}, \texttt{A1=MAX(B1 B2)}, \texttt{A1=MAX B1:B2)}, return \textbf{[0.0] (unrecognized cmd) >}
    \item Other invalid inputs can also be like
    \textttA{A4=AVG(A1:A3)xyz}, \texttt{A1=MAX(B1 B2)}, \texttt{A4=A5MIN(A3:A3)}, return \textbf{[0.0] (unrecognized cmd) >}
\end{enumerate}

\subsection{Cyclic Dependency}
If a formula creates a cyclic dependency (e.g., \texttt{A1=B1+1} and \texttt{B1=A1+1} or \texttt{A10=MAX(A5:A10)}), our program detects this and prevents infinite loops. The output will be:
\begin{center}
    \textbf{[0.0] (cycle detected) >}
\end{center}

\subsection{Division by Zero}
Any division by zero, such as \texttt{A1=B1/0}, results in an error. The cell value will be:
\begin{center}
    \textbf{ERR}
\end{center}
The program ensures that cell is assigned ERR till it is reassigned correctly instead of halting the program.

\subsection{Invalid Cells}
Accessing non-existent cells results in an error. For example:
\begin{enumerate}
    \item \texttt{A1=B10000+5} (when the sheet has only 100 rows)
    \item \texttt{X1=SUM(A1:A2)} (when column X does not exist)
    \item \texttt{scrollto ZZZ1000} 
\end{enumerate}
Such cases return:
\begin{center}
    \textbf{[0.0] (invalid cell) >}
\end{center}

\subsection{Incorrectly Running the Program}
Running the program with noninteger or out of range integer inputs in the row and column (e.g.nge (e.g. \texttt{./sheet six forty}) will result in:
\begin{center}
    \textbf{error: write valid number of rows and columns.}
\end{center}

\begin{figure}
\centering
\includegraphics[width=0.3\textwidth]{copass.jpg}
\caption{Example output of the spreadsheet program.}
\end{figure}

\section{STRUCTURE OF PROGRAM}

\begin{enumerate}
    \item Display.c
    \begin{itemize}
        \item int main(int argc, char *argv[])
        \item void display (cell ** mysheet, int R, int C, int x, int y)
        \item char * print\_string(int n, int padding)
    \end{itemize}
    \item Asscop.c:
    \begin{itemize}
        \item char parser(char* input)
        \item bool edgehandler (int cellhandle, int lhs, cell* lhscell, int extra, cell oldcell)
        \item void addDependencies (cell * lhscell, char op, int lhs)
        \item void deleteDependencies(cell *lhscell, int lhs)
        \item int cell\_handler(char *cell)
        \item bool is\_int (char *s)
    \end{itemize}
\end{enumerate}


\section{How to Add Lists}
Lists can be created using:
\begin{enumerate}
    \item Automatic numbering:
    \begin{enumerate}
        \item Like this,
        \item And like this.
    \end{enumerate}
    \item Bullet points:
    \begin{itemize}
        \item Like this,
        \item And like this.
    \end{itemize}
\end{enumerate}

\end{document}
