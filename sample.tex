\documentclass[a4paper]{article}

%% Language and font encodings
\usepackage[english]{babel}
\usepackage[T1]{fontenc}

%% Sets page size and margins
\usepackage[a4paper,top=3cm,bottom=2cm,left=3cm,right=3cm,marginparwidth=1.75cm]{geometry}

%% Useful packages
\usepackage{amsmath}
\usepackage{graphicx}
\usepackage[colorinlistoftodos]{todonotes}
\usepackage[colorlinks=true, allcolors=blue]{hyperref}

\title{COP290 C LAB: SPREADSHEET PROGRAM}
\author{Samarth Patel, Priyanka Sar, Rishika Garg}

\begin{document}
\maketitle

\begin{abstract}
Your abstract.
\end{abstract}

\section{DESIGN DECISIONS}

Our spreadsheet program consists of a makefile running which compiles our other files :- asscop.c, display.c, expcalc.c, graph.c, avltree.c, and helper.c to create an executable spreadsheet analogous sheet. This sheet can be fed formulas and control inputs. Display.c initializes our sheet with the required number of rows and columns with each cell=0. Then with the input, it calls the getline function (defined in helper.c) to read and scroll the sheet or disable/enable display. Alternatively, it calls the parser (defined in asscop.c) to do the necessary assignments and recalculations. Each cell in our spreadsheet is a structure that contains its value and some other values required for recalculations, whether the cell is visited in graph traversal, rows and columns of the cells it depends on and the roots of the two trees it is part of. We have used avltrees (avltree.c) to calculate the mix/max thing and to store the dependencies of each cell. 

\section{EDGE CASES AND ERROR SCENARIOS}

\subsection{How to add Comments}

Comments can be added to your project by clicking on the Review menu in the toolbar above. To reply to a comment, simply click the reply button in the lower right corner of the comment, and you can close them when you're done.

\subsection{How to include Figures}

First you have to upload the image file from your computer using the upload link in the project menu. Then use the includegraphics command to include it in your document. Use the figure environment and the caption command to add a number and a caption to your figure. See the code for Figure \ref{fig:frog} in this section for an example.

\begin{figure}
\centering
\includegraphics[width=0.3\textwidth]{frog.jpg}
\caption{\label{fig:frog}This frog was uploaded via the project menu.}
\end{figure}

\subsection{How to add Tables}

Use the table and tabular commands for basic tables --- see Table~\ref{tab:widgets}, for example. 

\begin{table}
\centering
\begin{tabular}{l|r}
Item & Quantity \\\hline
Widgets & 42 \\
Gadgets & 13
\end{tabular}
\caption{\label{tab:widgets}An example table.}
\end{table}

\subsection{How to write Mathematics}

\LaTeX{} is great at typesetting mathematics. Let $X_1, X_2, \ldots, X_n$ be a sequence of independent and identically distributed random variables with $\text{E}[X_i] = \mu$ and $\text{Var}[X_i] = \sigma^2 < \infty$, and let
\[S_n = \frac{X_1 + X_2 + \cdots + X_n}{n}
      = \frac{1}{n}\sum_{i}^{n} X_i\]
denote their mean. Then as $n$ approaches infinity, the random variables $\sqrt{n}(S_n - \mu)$ converge in distribution to a normal $\mathcal{N}(0, \sigma^2)$.


\subsection{How to create Sections and Subsections}

Use section and subsections to organize your document. Simply use the section and subsection buttons in the toolbar to create them, and we'll handle all the formatting and numbering automatically.

\subsection{How to add Lists}

You can make lists with automatic numbering \dots

\begin{enumerate}
\item Like this,
\item and like this.
\end{enumerate}
\dots or bullet points \dots
\begin{itemize}
\item Like this,
\item and like this.
\end{itemize}

\subsection{How to add Citations and a References List}

You can upload a \verb|.bib| file containing your BibTeX entries, created with JabRef; or import your \href{https://www.overleaf.com/blog/184}{Mendeley}, CiteULike or Zotero library as a \verb|.bib| file. You can then cite entries from it, like this: \cite{greenwade93}. Just remember to specify a bibliography style, as well as the filename of the \verb|.bib|.

You can find a \href{https://www.overleaf.com/help/97-how-to-include-a-bibliography-using-bibtex}{video tutorial here} to learn more about BibTeX.

We hope you find Overleaf useful, and please let us know if you have any feedback using the help menu --- or use the contact form at \url{https://www.overleaf.com/contact}!

\bibliographystyle{alpha}
\bibliography{sample}

\end{document}